% Options for packages loaded elsewhere
\PassOptionsToPackage{unicode}{hyperref}
\PassOptionsToPackage{hyphens}{url}
%
\documentclass[
]{article}
\usepackage{amsmath,amssymb}
\usepackage{iftex}
\ifPDFTeX
  \usepackage[T1]{fontenc}
  \usepackage[utf8]{inputenc}
  \usepackage{textcomp} % provide euro and other symbols
\else % if luatex or xetex
  \usepackage{unicode-math} % this also loads fontspec
  \defaultfontfeatures{Scale=MatchLowercase}
  \defaultfontfeatures[\rmfamily]{Ligatures=TeX,Scale=1}
\fi
\usepackage{lmodern}
\ifPDFTeX\else
  % xetex/luatex font selection
\fi
% Use upquote if available, for straight quotes in verbatim environments
\IfFileExists{upquote.sty}{\usepackage{upquote}}{}
\IfFileExists{microtype.sty}{% use microtype if available
  \usepackage[]{microtype}
  \UseMicrotypeSet[protrusion]{basicmath} % disable protrusion for tt fonts
}{}
\makeatletter
\@ifundefined{KOMAClassName}{% if non-KOMA class
  \IfFileExists{parskip.sty}{%
    \usepackage{parskip}
  }{% else
    \setlength{\parindent}{0pt}
    \setlength{\parskip}{6pt plus 2pt minus 1pt}}
}{% if KOMA class
  \KOMAoptions{parskip=half}}
\makeatother
\usepackage{xcolor}
\usepackage[margin=1in]{geometry}
\usepackage{longtable,booktabs,array}
\usepackage{calc} % for calculating minipage widths
% Correct order of tables after \paragraph or \subparagraph
\usepackage{etoolbox}
\makeatletter
\patchcmd\longtable{\par}{\if@noskipsec\mbox{}\fi\par}{}{}
\makeatother
% Allow footnotes in longtable head/foot
\IfFileExists{footnotehyper.sty}{\usepackage{footnotehyper}}{\usepackage{footnote}}
\makesavenoteenv{longtable}
\usepackage{graphicx}
\makeatletter
\def\maxwidth{\ifdim\Gin@nat@width>\linewidth\linewidth\else\Gin@nat@width\fi}
\def\maxheight{\ifdim\Gin@nat@height>\textheight\textheight\else\Gin@nat@height\fi}
\makeatother
% Scale images if necessary, so that they will not overflow the page
% margins by default, and it is still possible to overwrite the defaults
% using explicit options in \includegraphics[width, height, ...]{}
\setkeys{Gin}{width=\maxwidth,height=\maxheight,keepaspectratio}
% Set default figure placement to htbp
\makeatletter
\def\fps@figure{htbp}
\makeatother
\setlength{\emergencystretch}{3em} % prevent overfull lines
\providecommand{\tightlist}{%
  \setlength{\itemsep}{0pt}\setlength{\parskip}{0pt}}
\setcounter{secnumdepth}{5}

\usecolortheme{beaver}
\useinnertheme{rounded}
\useoutertheme[subsection=false,footline=authortitle]{miniframes}  

\beamertemplatenavigationsymbolsempty
\setbeamertemplate{headline}{}

\setbeamerfont{block title}{size={}}

\definecolor{kdisgreen}{RGB}{0, 99, 52}
\definecolor{kdisplatinum}{RGB}{167, 169, 172}

\setbeamercolor{alerted text}{fg=kdisgreen}
\setbeamercolor{example text}{fg=kdisplatinum}

\setbeamercolor*{palette secondary}{fg=white,bg=kdisplatinum} %subsection
\setbeamercolor*{palette tertiary}{fg=white,bg=kdisgreen} %section

\setbeamercolor{title}{fg=black}         %Title of presentation
\setbeamercolor{section title}{fg=white, bg=kdisgreen}
\setbeamercolor{subsection title}{fg=kdisgreen, bg=kdisplatinum}
\setbeamercolor{frametitle}{fg=kdisgreen, bg=kdisplatinum}

\setbeamerfont{frametitle}{size=\small}
\setbeamerfont{frametitle}{series=\bfseries}

\setbeamertemplate{blocks}[rounded][shadow=true] 
\setbeamertemplate{items}[triangle]
\setbeamercolor{item}{fg=gray, bg=white}

\AtBeginSection{}

\usepackage{tabularx}
\usepackage{changepage}
\usepackage{amsmath}
\usepackage{mathrsfs}
\usepackage{mathtools}
\usepackage{xcolor}
\usepackage{adjustbox,lipsum}
\usepackage{graphicx}
\usepackage{booktabs}
\usepackage{multirow}
\usepackage{appendixnumberbeamer} 



\ifLuaTeX
  \usepackage{selnolig}  % disable illegal ligatures
\fi
\IfFileExists{bookmark.sty}{\usepackage{bookmark}}{\usepackage{hyperref}}
\IfFileExists{xurl.sty}{\usepackage{xurl}}{} % add URL line breaks if available
\urlstyle{same}
\hypersetup{
  hidelinks,
  pdfcreator={LaTeX via pandoc}}

\title{Assignment 4}
\usepackage{etoolbox}
\makeatletter
\providecommand{\subtitle}[1]{% add subtitle to \maketitle
  \apptocmd{\@title}{\par {\large #1 \par}}{}{}
}
\makeatother
\subtitle{Ph.D.~Applied Microeconometrics\\
KDI School Fall 2023}
\author{}
\date{\vspace{-2.5em}2023-10-31}

\begin{document}
\maketitle

\textbf{Due date: Friday, November 10th at 11:59pm}

For this assignment, you will be continuing with the \texttt{castle.dta} data -- which I've put in the assignment folder -- the same data from last week.

Below is a list of tasks. I would like you to create a properly formated PDF file, as if it were a paper. This means that raw code should not appear in the PDF file. In a change, \emph{this week I would like you to format this more like a paper.} I want a short introduction (you can use \texttt{\#\ Introduction} in your R Markdown file), and then a separate section for each task. Each section should have a title, and then a discussion of the results. You do not need a conclusion section here.

Tasks:

\begin{enumerate}
\def\labelenumi{\arabic{enumi}.}
\tightlist
\item
  Recreate the two-way fixed effects estimates from your last assignment. Put them in a nice table. Discuss the results again.
\item
  Discuss the recent literature on (possible) bias in two-way fixed effects. What is the ``forbidden comparison'' we discussed in class?
\item
  Using a package of your choice (you might want to just use \texttt{did2s} from class), estimate the two-way fixed effects model. How do they compare to the results you got in 1.? Why do you think the results are different/similar (depending on what you got)?
\item
  Create an event study plot for the two-way fixed effects model. You may want to reference my code from class. Discuss the results.
\end{enumerate}

Make sure to \emph{interpret} your output. Do not just present regressions/figures/summary statistics.

As before, please submit the following files:

\begin{itemize}
\tightlist
\item
  Your R Markdown file
\item
  Your knitted PDF file
\item
  Any other scripts you used to complete the assignment
\end{itemize}

\end{document}
