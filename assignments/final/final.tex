% Options for packages loaded elsewhere
\PassOptionsToPackage{unicode}{hyperref}
\PassOptionsToPackage{hyphens}{url}
%
\documentclass[
]{article}
\usepackage{amsmath,amssymb}
\usepackage{iftex}
\ifPDFTeX
  \usepackage[T1]{fontenc}
  \usepackage[utf8]{inputenc}
  \usepackage{textcomp} % provide euro and other symbols
\else % if luatex or xetex
  \usepackage{unicode-math} % this also loads fontspec
  \defaultfontfeatures{Scale=MatchLowercase}
  \defaultfontfeatures[\rmfamily]{Ligatures=TeX,Scale=1}
\fi
\usepackage{lmodern}
\ifPDFTeX\else
  % xetex/luatex font selection
\fi
% Use upquote if available, for straight quotes in verbatim environments
\IfFileExists{upquote.sty}{\usepackage{upquote}}{}
\IfFileExists{microtype.sty}{% use microtype if available
  \usepackage[]{microtype}
  \UseMicrotypeSet[protrusion]{basicmath} % disable protrusion for tt fonts
}{}
\makeatletter
\@ifundefined{KOMAClassName}{% if non-KOMA class
  \IfFileExists{parskip.sty}{%
    \usepackage{parskip}
  }{% else
    \setlength{\parindent}{0pt}
    \setlength{\parskip}{6pt plus 2pt minus 1pt}}
}{% if KOMA class
  \KOMAoptions{parskip=half}}
\makeatother
\usepackage{xcolor}
\usepackage[margin=1in]{geometry}
\usepackage{longtable,booktabs,array}
\usepackage{calc} % for calculating minipage widths
% Correct order of tables after \paragraph or \subparagraph
\usepackage{etoolbox}
\makeatletter
\patchcmd\longtable{\par}{\if@noskipsec\mbox{}\fi\par}{}{}
\makeatother
% Allow footnotes in longtable head/foot
\IfFileExists{footnotehyper.sty}{\usepackage{footnotehyper}}{\usepackage{footnote}}
\makesavenoteenv{longtable}
\usepackage{graphicx}
\makeatletter
\def\maxwidth{\ifdim\Gin@nat@width>\linewidth\linewidth\else\Gin@nat@width\fi}
\def\maxheight{\ifdim\Gin@nat@height>\textheight\textheight\else\Gin@nat@height\fi}
\makeatother
% Scale images if necessary, so that they will not overflow the page
% margins by default, and it is still possible to overwrite the defaults
% using explicit options in \includegraphics[width, height, ...]{}
\setkeys{Gin}{width=\maxwidth,height=\maxheight,keepaspectratio}
% Set default figure placement to htbp
\makeatletter
\def\fps@figure{htbp}
\makeatother
\setlength{\emergencystretch}{3em} % prevent overfull lines
\providecommand{\tightlist}{%
  \setlength{\itemsep}{0pt}\setlength{\parskip}{0pt}}
\setcounter{secnumdepth}{5}

\usecolortheme{beaver}
\useinnertheme{rounded}
\useoutertheme[subsection=false,footline=authortitle]{miniframes}  

\beamertemplatenavigationsymbolsempty
\setbeamertemplate{headline}{}

\setbeamerfont{block title}{size={}}

\definecolor{kdisgreen}{RGB}{0, 99, 52}
\definecolor{kdisplatinum}{RGB}{167, 169, 172}

\setbeamercolor{alerted text}{fg=kdisgreen}
\setbeamercolor{example text}{fg=kdisplatinum}

\setbeamercolor*{palette secondary}{fg=white,bg=kdisplatinum} %subsection
\setbeamercolor*{palette tertiary}{fg=white,bg=kdisgreen} %section

\setbeamercolor{title}{fg=black}         %Title of presentation
\setbeamercolor{section title}{fg=white, bg=kdisgreen}
\setbeamercolor{subsection title}{fg=kdisgreen, bg=kdisplatinum}
\setbeamercolor{frametitle}{fg=kdisgreen, bg=kdisplatinum}

\setbeamerfont{frametitle}{size=\small}
\setbeamerfont{frametitle}{series=\bfseries}

\setbeamertemplate{blocks}[rounded][shadow=true] 
\setbeamertemplate{items}[triangle]
\setbeamercolor{item}{fg=gray, bg=white}

\AtBeginSection{}

\usepackage{tabularx}
\usepackage{changepage}
\usepackage{amsmath}
\usepackage{mathrsfs}
\usepackage{mathtools}
\usepackage{xcolor}
\usepackage{adjustbox,lipsum}
\usepackage{graphicx}
\usepackage{booktabs}
\usepackage{multirow}
\usepackage{appendixnumberbeamer} 



\ifLuaTeX
  \usepackage{selnolig}  % disable illegal ligatures
\fi
\usepackage{bookmark}
\IfFileExists{xurl.sty}{\usepackage{xurl}}{} % add URL line breaks if available
\urlstyle{same}
\hypersetup{
  hidelinks,
  pdfcreator={LaTeX via pandoc}}

\title{Final Exam}
\usepackage{etoolbox}
\makeatletter
\providecommand{\subtitle}[1]{% add subtitle to \maketitle
  \apptocmd{\@title}{\par {\large #1 \par}}{}{}
}
\makeatother
\subtitle{Ph.D.~Applied Microeconometrics\\
KDI School Fall 2024}
\author{}
\date{\vspace{-2.5em}2024-11-29}

\begin{document}
\maketitle

\textbf{Due date: Thursday, December 5th at 11:59pm}

Please work by yourself. As before, please submit the following files on eKDIS:

\begin{itemize}
\tightlist
\item
  Your R Markdown file
\item
  Your knitted PDF file
\item
  Any other scripts you used to complete the assignment
\end{itemize}

I would like all of your answers (including the data section) to be in a single markdown file. However, you are welcome to use another script for any of the analyses, if you would prefer. If you do, please include the script in your submission.

\section{Part 1: Short answer}\label{part-1-short-answer}

\textbf{Question 1}

We spent quite a bit of time discussing recent research on two-way fixed effects (TWFE). Discuss the following:

\begin{itemize}
\tightlist
\item
  What is the main problem with two-way fixed effects?

  \begin{itemize}
  \tightlist
  \item
    When does this problem arise?
  \end{itemize}
\item
  What are some of the solutions that have been proposed?
\item
  Do you agree with the following statement? Why or why not? ``Recent advancements in this area are the most important advancements in applied microeconometrics in the last 10 years.''
\end{itemize}

\textbf{Question 2}

Consider the following production function:
\begin{gather} \label{eq:prod} log(y_{it}) = \alpha_i + \beta_t + \gamma log(L_{it}) + \delta log(K_{it}) + \phi\left(log(L_{it})\times log(K_{it})\right) + \epsilon_{it} \end{gather}
where \(y_{it}\) is output, \(L_{it}\) is labor, \(K_{it}\) is capital, and \(\alpha_i\) and \(\beta_t\) are firm and time fixed effects, respectively. I want to estimate the marginal product of labor, which is given as:
\begin{gather} \label{eq:mp} \frac{\partial y_{it}}{\partial L_{it}} = \frac{y_{it}}{L_{it}}\left(\gamma + \phi log(K_{it})\right) \end{gather}

\begin{itemize}
\tightlist
\item
  What are the identification assumptions for estimating \(\gamma\)? (You can ignore the recent literature on TWFE.)
\item
  For estimating the production function itself, how would you deal with standard errors? Why?
\item
  Suppose I want to test the hypothesis that \(\gamma=\delta=\phi=0\).

  \begin{itemize}
  \tightlist
  \item
    What is the appropriate test statistic?
  \item
    How would you calculate the test statistic? (I mean by hand, not by using a canned function.)
  \end{itemize}
\item
  How would you calculate a confidence interval for the marginal product?
\end{itemize}

\textbf{Question 3}

A commonly taught assumption of OLS is that the error term is normally distributed. Discuss when this assumption is particularly important and when it is not. (Hint: think about the CLT.)

\section{Part 2: Getting your hands dirty with data}\label{part-2-getting-your-hands-dirty-with-data}

The dataset ``pollution.csv'' is subset of data from my pollution paper. The variables are:

\begin{itemize}
\tightlist
\item
  shrid: village identifier (this is panel data)
\item
  year: year
\item
  distfe: district identifier
\item
  pm25: pollution concentration (PM 2.5) during the growing season, logged
\item
  wind: the number of days (in 10s) in the season in which the village is downwind from a pollution source
\item
  yield: agricultural yield (in kilograms per hectare)
\item
  rain\_z: rainfall (z score) during the growing season
\item
  temp\_mean: temperature during the growing season
\end{itemize}

I do not specify how to deal with standard errors. I will leave that up to you.

\textbf{Question 4}

Consider the following regression:
\begin{gather} \label{eq:reg} log(yield_{it}) = \alpha_i + \gamma_t + \beta_1 pm25_{it} + \beta_2 rain_{it} + \beta_3 temp_{it} + \epsilon_{it}, \end{gather}
where \(\alpha_i\) is village fixed effects and \(\gamma_t\) is year fixed effects.

\begin{itemize}
\tightlist
\item
  Estimate the regression and output the results in a table.
\item
  Interpret the coefficient on \(pm25\).
\item
  What are the requirements for the coefficient on \(pm25\) to be interpreted as the causal effect of pollution on yield?

  \begin{itemize}
  \tightlist
  \item
    Do you think these requirements are satisfied? Why or why not?
  \end{itemize}
\item
  How did you specify the calculation of the variance-covariance matrix (standard errors)? Why?
\end{itemize}

\textbf{Question 5}

Consider the following two-stage least squares (2SLS) set up:
\begin{gather}  \label{eq:iv1} pm25_{it} = \alpha_i + \gamma_t + \beta_1 wind_{it} + \beta_2 rain_{it} + \beta_3 temp_{it} + \epsilon_{it} \\
                \label{eq:iv2} log(yield_{it}) = \phi_i + \psi_t + \delta_1 pm25_{it} + \delta_2 rain_{it} + \delta_3 temp_{it} + \eta_{it} \end{gather}

\begin{itemize}
\tightlist
\item
  Estimate the \emph{reduced form} regression and output the results in a table. Interpret the coefficient on \(wind\).
\item
  Estimate the first and second stages of the 2SLS regression and output the results in a table. Interpret the coefficient on \(pm25\).
\item
  What are the requirements for the coefficient on \(pm25\) to be interpreted as the causal effect of pollution on yield?

  \begin{itemize}
  \tightlist
  \item
    Do you think these requirements are satisfied? Why or why not? (You can refer to my paper if you'd like.)
  \end{itemize}
\end{itemize}

\subsection{Question 6}\label{question-6}

Suppose you want to use \emph{randomization inference} to test the hypothesis that the coefficient on \(pm25\) in equation \eqref{eq:iv2} is equal to zero. To do this, you need to give each village a \emph{random} treatment assignment from the rest of the villages. In other words, you need to take a randomly selected combination of \{wind, pm\} and assign it to each village (the randomly selected combination should come from the same village). The steps are as follows:

\begin{enumerate}
\def\labelenumi{\arabic{enumi}.}
\tightlist
\item
  Set a seed to ensure that your results are replicable.
\item
  Randomly select a combination of \{wind, pm\} and assign it to each village. However, you need to randomly select \emph{from the same year}.
\item
  Estimate the 2SLS regression using the randomly assigned IV and treatment variable.
\item
  Save the coefficient on \(pm25\) from the second stage (\(\delta_1\)).
\item
  Repeat this process 500 times.
\end{enumerate}

\begin{itemize}
\tightlist
\item
  Do steps 1 through 5.
\item
  Create a figure that shows the distribution of \(\delta_1\) from the 500 iterations.
\item
  Compare the distribution of treatment effects to the actual treatment effect from the 2SLS regression. What do you conclude?

  \begin{itemize}
  \tightlist
  \item
    In other words, interpret the output and what this teaches us about the estimated treatment effect.
  \end{itemize}
\end{itemize}

\end{document}
