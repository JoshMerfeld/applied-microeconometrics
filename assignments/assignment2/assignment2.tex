% Options for packages loaded elsewhere
\PassOptionsToPackage{unicode}{hyperref}
\PassOptionsToPackage{hyphens}{url}
%
\documentclass[
]{article}
\usepackage{amsmath,amssymb}
\usepackage{iftex}
\ifPDFTeX
  \usepackage[T1]{fontenc}
  \usepackage[utf8]{inputenc}
  \usepackage{textcomp} % provide euro and other symbols
\else % if luatex or xetex
  \usepackage{unicode-math} % this also loads fontspec
  \defaultfontfeatures{Scale=MatchLowercase}
  \defaultfontfeatures[\rmfamily]{Ligatures=TeX,Scale=1}
\fi
\usepackage{lmodern}
\ifPDFTeX\else
  % xetex/luatex font selection
\fi
% Use upquote if available, for straight quotes in verbatim environments
\IfFileExists{upquote.sty}{\usepackage{upquote}}{}
\IfFileExists{microtype.sty}{% use microtype if available
  \usepackage[]{microtype}
  \UseMicrotypeSet[protrusion]{basicmath} % disable protrusion for tt fonts
}{}
\makeatletter
\@ifundefined{KOMAClassName}{% if non-KOMA class
  \IfFileExists{parskip.sty}{%
    \usepackage{parskip}
  }{% else
    \setlength{\parindent}{0pt}
    \setlength{\parskip}{6pt plus 2pt minus 1pt}}
}{% if KOMA class
  \KOMAoptions{parskip=half}}
\makeatother
\usepackage{xcolor}
\usepackage[margin=1in]{geometry}
\usepackage{longtable,booktabs,array}
\usepackage{calc} % for calculating minipage widths
% Correct order of tables after \paragraph or \subparagraph
\usepackage{etoolbox}
\makeatletter
\patchcmd\longtable{\par}{\if@noskipsec\mbox{}\fi\par}{}{}
\makeatother
% Allow footnotes in longtable head/foot
\IfFileExists{footnotehyper.sty}{\usepackage{footnotehyper}}{\usepackage{footnote}}
\makesavenoteenv{longtable}
\usepackage{graphicx}
\makeatletter
\def\maxwidth{\ifdim\Gin@nat@width>\linewidth\linewidth\else\Gin@nat@width\fi}
\def\maxheight{\ifdim\Gin@nat@height>\textheight\textheight\else\Gin@nat@height\fi}
\makeatother
% Scale images if necessary, so that they will not overflow the page
% margins by default, and it is still possible to overwrite the defaults
% using explicit options in \includegraphics[width, height, ...]{}
\setkeys{Gin}{width=\maxwidth,height=\maxheight,keepaspectratio}
% Set default figure placement to htbp
\makeatletter
\def\fps@figure{htbp}
\makeatother
\setlength{\emergencystretch}{3em} % prevent overfull lines
\providecommand{\tightlist}{%
  \setlength{\itemsep}{0pt}\setlength{\parskip}{0pt}}
\setcounter{secnumdepth}{5}

\usecolortheme{beaver}
\useinnertheme{rounded}
\useoutertheme[subsection=false,footline=authortitle]{miniframes}  

\beamertemplatenavigationsymbolsempty
\setbeamertemplate{headline}{}

\setbeamerfont{block title}{size={}}

\definecolor{kdisgreen}{RGB}{0, 99, 52}
\definecolor{kdisplatinum}{RGB}{167, 169, 172}

\setbeamercolor{alerted text}{fg=kdisgreen}
\setbeamercolor{example text}{fg=kdisplatinum}

\setbeamercolor*{palette secondary}{fg=white,bg=kdisplatinum} %subsection
\setbeamercolor*{palette tertiary}{fg=white,bg=kdisgreen} %section

\setbeamercolor{title}{fg=black}         %Title of presentation
\setbeamercolor{section title}{fg=white, bg=kdisgreen}
\setbeamercolor{subsection title}{fg=kdisgreen, bg=kdisplatinum}
\setbeamercolor{frametitle}{fg=kdisgreen, bg=kdisplatinum}

\setbeamerfont{frametitle}{size=\small}
\setbeamerfont{frametitle}{series=\bfseries}

\setbeamertemplate{blocks}[rounded][shadow=true] 
\setbeamertemplate{items}[triangle]
\setbeamercolor{item}{fg=gray, bg=white}

\AtBeginSection{}

\usepackage{tabularx}
\usepackage{changepage}
\usepackage{amsmath}
\usepackage{mathrsfs}
\usepackage{mathtools}
\usepackage{xcolor}
\usepackage{adjustbox,lipsum}
\usepackage{graphicx}
\usepackage{booktabs}
\usepackage{multirow}
\usepackage{appendixnumberbeamer} 



\ifLuaTeX
  \usepackage{selnolig}  % disable illegal ligatures
\fi
\usepackage{bookmark}
\IfFileExists{xurl.sty}{\usepackage{xurl}}{} % add URL line breaks if available
\urlstyle{same}
\hypersetup{
  hidelinks,
  pdfcreator={LaTeX via pandoc}}

\title{Assignment 2}
\usepackage{etoolbox}
\makeatletter
\providecommand{\subtitle}[1]{% add subtitle to \maketitle
  \apptocmd{\@title}{\par {\large #1 \par}}{}{}
}
\makeatother
\subtitle{Ph.D.~Applied Microeconometrics\\
KDI School Fall 2023}
\author{}
\date{\vspace{-2.5em}2024-10-01}

\begin{document}
\maketitle

\textbf{Due date: Monday, October 14th at 6:59pm}

For this assignment, you will be using the \texttt{village\_economiccensus.rds} data in the assignment folder. I have done a bit of cleaning for you, including a variable \texttt{year} that is the year of the census. I have renamed all of the economic census variables to drop the \texttt{ec\_} prefix. You can find the definition of the variables on the SHRUG website, from which I downloaded the data. The data dictionary for the economic census is \href{https://docs.devdatalab.org/SHRUG-Metadata/Economic\%20Census/Tables/ec90-metadata/}{\textcolor{kdisgreen}{here}}.

The dataset is in an R-specific format. You can load it into R using the \texttt{readRDS()} function.

Below is a list of tasks. I would like you to create a properly formatted PDF file, as if it were a paper. This means that raw code should not appear in the PDF file. When you estimate a regression, the choice of standard errors is up to you. However, I'd like you to justify your choice of standard errors in each case. For 3. and 4., please treat the data as a simple cross-section.

Tasks:

\begin{enumerate}
\def\labelenumi{\arabic{enumi}.}
\tightlist
\item
  Create some new variables:

  \begin{itemize}
  \tightlist
  \item
    Proportion of total employment that is women
  \item
    Proportion of total employment that is government employment
  \end{itemize}
\item
  Create a table that shows the mean of these variables by year.
\item
  Create a figure of your choice. The choice of x and y variables are completely up to you. Describe the figure and interpret it.
\item
  Using OLS, estimate a regression (and present the output) that allows you to extract the exact same information as the information in the table you created in step 2. Please make sure to interpret the output.
\item
  Estimate a poisson regression that shows the relationship between the total number of female employees (on the left-hand side) and the following variables on the right-hand side: total employment, proportion of employment that is government employment, and year. Present the output of this regression. Please make sure to interpret the output.
\item
  The data also has an additional variable in it, called \texttt{shrid2}. This variable is a village identifier. It turns out that the data I've given you is \emph{panel data}, with the exact same villages observed across four separate years. Re-estimate your regressions in 3. and 4. using this new information. How does this change any of the decisions you made in 3. and 4.? Please make sure to interpret the output and explain any changes you made.
\item
  Now, estimate a regression that shows the relationship between the proportion of jobs that are government jobs (on the left-hand side) and the following variables on the right-hand side:

  \begin{itemize}
  \tightlist
  \item
    Whether there is a coal plant within 50 kilometers
  \item
    Whether there is a coal plant within 100 kilometers
  \item
    Total employment (count)
  \end{itemize}
\end{enumerate}

I am interested in how much larger the effect within 50 km is than the effect within 100km is. In other words, if our regression equation is
\[ y_{i} = \beta_0 + \beta_1x_{within50} + \beta_2x_{within100} + \beta_3x_{emp} + \epsilon_i, \]
I am interested in the ratio \(\frac{\beta_1}{\beta_2}\). Please bootstrap the standard errors of this ratio. Please present the output of this regression and the standard errors. I want two separate standard errors:

\begin{itemize}
\tightlist
\item
  One assuming there is no clustering on \texttt{shrid}
\item
  One assuming there is clustering on \texttt{shrid}
\end{itemize}

As before, please submit the following files:

\begin{itemize}
\tightlist
\item
  Your R Markdown file
\item
  Your knitted PDF file
\item
  Any other scripts you used to complete the assignment
\end{itemize}

\end{document}
